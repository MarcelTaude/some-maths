%% prelude.tex
%%

%% \documentclass[10pt]{article} % use larger type; default would be 10pt
\documentclass[psamsfonts]{amsart} % use larger type; default would be 10pt
%% \documentclass{amsart} % use larger type; default would be 10pt

%% \usepackage[frenchb]{babel}
\usepackage[utf8]{inputenc} % set input encoding (not needed with XeLaTeX)
\usepackage[T1]{fontenc}

\usepackage{amsmath}

%%% Examples of Article customizations
% These packages are optional, depending whether you want the features they provide.
% See the LaTeX Companion or other references for full information.

%%% PAGE DIMENSIONS
\usepackage{geometry} % to change the page dimensions
\geometry{a4paper} % or letterpaper (US) or a5paper or....

%% not compatible with AMS
%% \usepackage{graphicx} % support the \includegraphics command and options

\usepackage[parfill]{parskip} % Activate to begin paragraphs with an empty line rather than an indent

%%% PACKAGES
%% \usepackage{booktabs} % for much better looking tables
%% \usepackage{array} % for better arrays (eg matrices) in maths
%% \usepackage{paralist} % very flexible & customisable lists (eg. enumerate/itemize, etc.)
%% \usepackage{verbatim} % adds environment for commenting out blocks of text & for better verbatim
%% \usepackage{subfig} % make it possible to include more than one captioned figure/table in a single float
% These packages are all incorporated in the memoir class to one degree or another...

%%% HEADERS & FOOTERS
%% \usepackage{fancyhdr} % This should be set AFTER setting up the page geometry
%% \pagestyle{fancy} % options: empty , plain , fancy
%% \renewcommand{\headrulewidth}{0pt} % customise the layout...
%% \lhead{}\chead{}\rhead{}
%% \lfoot{}\cfoot{\thepage}\rfoot{}


%%% commandes personnelles
\usepackage{listings}
\usepackage{xspace}
\usepackage{color}

\definecolor{dgreen}{rgb}{0,0.5,0}
\usepackage[colorlinks=true,linkcolor=dgreen,urlcolor=blue]{hyperref}
\usepackage{multicol}

%% formattage de la table des matières : indentation en fonction du niveau
\setcounter{tocdepth}{3}% to get subsubsections in toc

\let\oldtocsection=\tocsection

\let\oldtocsubsection=\tocsubsection

\let\oldtocsubsubsection=\tocsubsubsection

\renewcommand{\tocsection}[2]{\hspace{0em}\oldtocsection{#1}{#2}}
\renewcommand{\tocsubsection}[2]{\hspace{1em}\oldtocsubsection{#1}{#2}}
\renewcommand{\tocsubsubsection}[2]{\hspace{2em}\oldtocsubsubsection{#1}{#2}}

