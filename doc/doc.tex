%% doc.tex

%% prelude.tex
%%

%% \documentclass[10pt]{article} % use larger type; default would be 10pt
\documentclass[psamsfonts]{amsart} % use larger type; default would be 10pt
%% \documentclass{amsart} % use larger type; default would be 10pt
\pagestyle{headings}

%% \usepackage[frenchb]{babel}
\usepackage[utf8]{inputenc} % set input encoding (not needed with XeLaTeX)
\usepackage[T1]{fontenc}

%% \usepackage{amsmath}
\usepackage{amsmath,amsthm}

%%% Examples of Article customizations
% These packages are optional, depending whether you want the features they provide.
% See the LaTeX Companion or other references for full information.

%%% PAGE DIMENSIONS
\usepackage{geometry} % to change the page dimensions
\geometry{a4paper} % or letterpaper (US) or a5paper or....

%% not compatible with AMS
%% \usepackage{graphicx} % support the \includegraphics command and options

\usepackage[parfill]{parskip} % Activate to begin paragraphs with an empty line rather than an indent

%%% PACKAGES
%% \usepackage{booktabs} % for much better looking tables
%% \usepackage{array} % for better arrays (eg matrices) in maths
%% \usepackage{paralist} % very flexible & customisable lists (eg. enumerate/itemize, etc.)
%% \usepackage{verbatim} % adds environment for commenting out blocks of text & for better verbatim
%% \usepackage{subfig} % make it possible to include more than one captioned figure/table in a single float
% These packages are all incorporated in the memoir class to one degree or another...

%%% HEADERS & FOOTERS
%% \usepackage{fancyhdr} % This should be set AFTER setting up the page geometry
%% \pagestyle{fancy} % options: empty , plain , fancy
%% \renewcommand{\headrulewidth}{0pt} % customise the layout...
%% \lhead{}\chead{}\rhead{}
%% \lfoot{}\cfoot{\thepage}\rfoot{}


%%% commandes personnelles
\usepackage{listings}
\usepackage{xspace}
\usepackage{color}

\definecolor{dgreen}{rgb}{0,0.5,0}
\usepackage[colorlinks=true,linkcolor=dgreen,urlcolor=blue]{hyperref}
\usepackage{multicol}

%% formattage de la table des matières : indentation en fonction du niveau
\setcounter{tocdepth}{3}% to get subsubsections in toc

\let\oldtocsection=\tocsection

\let\oldtocsubsection=\tocsubsection

\let\oldtocsubsubsection=\tocsubsubsection

\renewcommand{\tocsection}[2]{\hspace{0em}\oldtocsection{#1}{#2}}
\renewcommand{\tocsubsection}[2]{\hspace{1em}\oldtocsubsection{#1}{#2}}
\renewcommand{\tocsubsubsection}[2]{\hspace{2em}\oldtocsubsubsection{#1}{#2}}



\begin{document}

\title{too old to demonstrate, to young to die}
\author{Marcel Taude}
\date{November 2015}

\begin{abstract}
	Playing with integer equations is really nice. This paper is a ... hmmm ... demonstration of an old adage \emph{if you don't practice mathematics for a long time, it's hard to come back to it}. Even when listening cool jazz from \emph{Lee Konitz}. At least, I played with \LaTeX\ and \emph{gcc}.

	At first, I wanted to find an algorithm to find integers which verify $a^3 + b^3 + c^3 = 1000^2.a + 1000.b + c$ with $a, b, c \in [0, 1000[$. There is an obvious lazy algorithm which tries every values of $a$, $b$ and $c$ in the range $[0, 1000[$. I wanted a clever one, not sure I found it, but read.
\end{abstract}	

\maketitle

\tableofcontents

\lstset{basicstyle=\small\ttfamily,language=make,morekeywords={ifeq,ifneq,ifdef,ifndef,else,endif,addprefix,notdir,PHONY}, keywordstyle=\bfseries\color{kword},backgroundcolor=\color{lgray},breaklines=true}


\section{the problem}
Find all integers which verify :
\begin{equation}\label{eq:start}
	a^3 + b^3 + c^3 = 1000^2.a + 1000.b + c \text{, with } a, b, c \in [0, 1000[
\end{equation}
We can \emph{easily} find all the integers $(a, b, c)$ verfiying \eqref{eq:start} by computing the two members of \eqref{eq:start} for every values $(a, b, c)$ from 0 to 999. Can we find a more clever way to get these values?

We can transform \eqref{eq:start} in :
\begin{equation}\label{eq:second}
\begin{split}
	c^3 - c &= 1000^2.a + 1000.b - a^3 - b^3 \\
	c(c - 1)(c + 1) &= a(1000 - a)(1000 + a) + b(1000 - b^2)
 \end{split}
\end{equation}

It becomes more obvious that the triplets $(0, 0, 0)$ and $(0, 0, 1)$ are two solutions of \eqref{eq:start}.

\section{... and this is the place where I became mad}
So much time without playing with mathematics makes things harder than I thought. Music helps me to keep cool, some good old jazz played by \emph{Lee Konitz} in the late fifties, and that's all. At least, I am playing with \LaTeX and \emph{gcc}...


\section{reboot}
We will do it in a lighter way. We started with power 3, bad idea. 

\begin{equation}\label{eq:pow2}
	a^2 + b^2 = 100.a + b \text{ with } a, b \in [0, 100[
\end{equation}

We can make this transformation :
\begin{equation}\label{eq:pow2bis}
	b(b-1) = a(100 - a)
\end{equation}

We can see that the left side of \eqref{eq:pow2ter} must be even, so $a$ is even too.

\subsection{the even side}
We begin with the case of $b$ even :
\begin{equation}\label{eq:pow2ter}
\begin{split}
	a &= 2 \alpha \\
	b &= 2 \beta \\
	2 \beta(2 \beta - 1) &= 2 \alpha(100 - 2 \alpha) \\
	\beta(2\beta - 1) &= 2 \alpha(50 - \alpha)
\end{split}
\end{equation}

If so, $\beta$ must be even, \emph{i.e.} :
\begin{equation}\label{eq:pow2:4}
\begin{split}
	\beta &= 2 \gamma \\
	2 \gamma (4 \gamma - 1) &= 2 \alpha (50 - \alpha) \\
	\gamma (4 \gamma - 1) &= \alpha (50 - \alpha)
\end{split}
\end{equation}

The good news is : 
\begin{equation}\label{eq:pow2:5}
\begin{split}
	\gamma &\in [0, 25[ \\
	\alpha &\in [0, 50[ \\
	a &= 2 \alpha \\
	b &= 4 \gamma
\end{split}
\end{equation}

We have less couples $(\alpha, \gamma)$ to test.

\subsection{the odd side}
Now, we are looking at odd values of $b$:
\begin{equation}\label{eq:pow2odd}
\begin{split}
	a &= 2 \alpha \\
	b &= 2 \beta + 1\\
	2 \beta(2 \beta + 1) &= 2 \alpha(100 - 2 \alpha) \\
	\beta(2\beta + 1) &= 2 \alpha(50 - \alpha)
\end{split}
\end{equation}

We comme back to the previous case with few changes :
\begin{equation}\label{eq:pow2:5}
\begin{split}
	\gamma &\in [0, 25] \\
	\alpha &\in [0, 50[ \\
	a &= 2 \alpha \\
	b &= 4 \gamma + 1
\end{split}
\end{equation}

\subsection{... and now}
It's time to write some code to test all these beautiful formulas. For this first lines of code, I
choose the \emph{C} language, the one I practice the most. Later I will use other languages like \emph{F\#}.




\section{notes}
I wrote this document with \LaTeX{}, using the AMS packages (\emph{cf.} \cite{bib:amsinstr} and \cite{bib:amsusersg}). The \emph{href} package (\emph{cf.} \cite{bib:hrefpack}) gives me the links.

\begin{thebibliography}{ABCDEFG12334567890}
\bibitem[AMSInstr2004]{bib:amsinstr}
\emph{Instructions for Preparation of Papers and Monographs}, AMS \\
\url{ftp://ftp.ams.org/pub/author-info/documentation/amslatex/instr-l.pdf} (2004)

\bibitem[AMSUsersG2002]{bib:amsusersg}
\emph{User's Guide for the amsmath Package}, AMS \\
\url{ftp://ftp.ams.org/pub/tex/doc/amsmath/amsldoc.pdf} (1999--2002)

\bibitem[HREFPackage]{bib:hrefpack}
\emph{LaTeX/Hyperlinks}, Wikipedia \\
\url{https://en.wikibooks.org/wiki/LaTeX/Hyperlinks#Commands}

\end{thebibliography}

\end{document}
