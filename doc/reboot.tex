
\section{reboot}
We will do it in a lighter way. We started with power 3, bad idea. 

\begin{equation}\label{eq:pow2}
	a^2 + b^2 = 100.a + b \text{ with } a, b \in [0, 100[
\end{equation}

We can make this transformation :
\begin{equation}\label{eq:pow2bis}
	b(b-1) = a(100 - a)
\end{equation}

We can see that the left side of \eqref{eq:pow2ter} must be even, so $a$ is even too.

\subsection{the even side}
We begin with the case of $b$ even :
\begin{equation}\label{eq:pow2ter}
\begin{split}
	a &= 2 \alpha \\
	b &= 2 \beta \\
	2 \beta(2 \beta - 1) &= 2 \alpha(100 - 2 \alpha) \\
	\beta(2\beta - 1) &= 2 \alpha(50 - \alpha)
\end{split}
\end{equation}

If so, $\beta$ must be even, \emph{i.e.} :
\begin{equation}\label{eq:pow2:4}
\begin{split}
	\beta &= 2 \gamma \\
	2 \gamma (4 \gamma - 1) &= 2 \alpha (50 - \alpha) \\
	\gamma (4 \gamma - 1) &= \alpha (50 - \alpha)
\end{split}
\end{equation}

The good news is : 
\begin{equation}\label{eq:pow2:5}
\begin{split}
	\gamma &\in [0, 25[ \\
	\alpha &\in [0, 50[ \\
	a &= 2 \alpha \\
	b &= 4 \gamma
\end{split}
\end{equation}

We have less couples $(\alpha, \gamma)$ to test.

\subsection{the odd side}
Now, we are looking at odd values of $b$:
\begin{equation}\label{eq:pow2odd}
\begin{split}
	a &= 2 \alpha \\
	b &= 2 \beta + 1\\
	2 \beta(2 \beta + 1) &= 2 \alpha(100 - 2 \alpha) \\
	\beta(2\beta + 1) &= 2 \alpha(50 - \alpha)
\end{split}
\end{equation}

We comme back to the previous case with few changes :
\begin{equation}\label{eq:pow2:5}
\begin{split}
	\gamma &\in [0, 25] \\
	\alpha &\in [0, 50[ \\
	a &= 2 \alpha \\
	b &= 4 \gamma + 1
\end{split}
\end{equation}

\subsection{... and now}
It's time to write some code to test all these beautiful formulas. For this first lines of code, I
choose the \emph{C} language, the one I practice the most. Later I will use other languages like \emph{F\#}.

